\documentclass[bachelor,zhspacing]{cqu}  %单面打印版本
\usepackage{etex}
\def\tightlist{}

%%在这增加你需要的其它包
\definecolor{hellgelb}{rgb}{1,1,0.8}
\definecolor{colKeys}{rgb}{0,0,1}
\definecolor{colIdentifier}{rgb}{0,0,0}
\definecolor{colComments}{rgb}{1,0,0}
\definecolor{colString}{rgb}{0,0.5,0}
\usepackage{listings}
\lstset{%
    float=hbp,%
    basicstyle=\ttfamily\small, %
    identifierstyle=\color{colIdentifier}, %
    keywordstyle=\color{colKeys}, %
    stringstyle=\color{colString}, %
    commentstyle=\color{colComments}, %
    columns=flexible, %
    tabsize=4, %
    frame=single, %
    extendedchars=true, %
    showspaces=false, %
    showstringspaces=false, %
    numbers=left, %
    numberstyle=\tiny, %
    breaklines=true, %
   backgroundcolor=\color{hellgelb}, %
    breakautoindent=true, %
    captionpos=b,%
	xleftmargin=0pt%
}

\begin{document}

%-----------------------------------论文题目-------------------------------------------------
\xuehao{20121892}
\cntitle{重庆大学\LaTeX 学位论文模板CQU使用说明}
\cnauthor{张乐}
\cnmajor{软件工程}
\cnteacher{葛永新}
\cnxueyuan{软件学院}
\entitle{An Instruction of the \LaTeX\ Templet for Chongqing University Thesis}
\enauthor{Le Zhang}
\enmajor{Software Engineering}
\enteacher{Prof. Yongxin Ge}
\enxueyuan{College of Software}
\cnkind{****}
\enkind{****}
%\cnzlteacher{ }  %%助理教师,如果必要,还要将cqu.cls中的有关该项前的%号去掉
%\enzlteacher{ }
\cndate{二O一六年六月}
\endate{June 2016}
%%%%只需修改上面的相关信息%%%%%%%%
\makecntitle 
\makeentitle 
%%%%%%%%%%%%%%%%%%%%%%%%%%%

\pagenumbering{Roman}
\setcounter{page}{0}
%------------------------------------文章摘要------------------------------------------------------------
\cnkeywords{模板,摘要,论文,\LaTeX}
\begin{cnabstract}
摘要是设计或论文内容不加注释和评论的简短陈述,应以第三人称陈述。它应具有独立性和自含性,
即不阅读设计或论文的全文,就能获得必要的信息,摘要的内容应包含与设计或论文同等量的主要信
息,供读者确定有无必要阅读全文,也供文摘等二次文献采用。\par
摘要一般应说明研究工作目的、实验研究方法、结果和最终结论等,而重点是结果和结论。摘要中一
般不用图、表、化学结构式、计算机程序,不用非公知公用的符号、术语和非法定的计量单位。\par
摘要页置于英文题名页后。 \par
中文摘要一般为400汉字左右,用小四号宋体。 \par
关键词是为了文献标引工作从设计(论文)中选取出来用以表示全文主题内容信息款目的单词或术
语。一般每篇设计(论文)应选取3~5个词作为关键词,关键词间用逗号隔开,最后一个词后不打标点
符号。以显著的字符排在同种语言摘要的下方。如有可能,尽量用《汉语主题词表》等词表提供的规
范词。\par
本文介绍重庆大学论文模板cqu的使用方法。本模板符合学校的本科论文格式基本要求,而硕博模板
有待完善。
本文的创新点主要有:
\begin{itemize*}
\item 用例子来解释模板的使用方法;
\item 用废话来填充无关紧要的部分;
\item 一边学习摸索一边编写新代码。
\end{itemize*}
(模板作者注:中文关键词定义cnkeywords应在使用中文摘要环境之前。英文关键词同理。)
\end{cnabstract} 
\enkeywords{template, \LaTeX, abstract, paper}
\begin{enabstract}
     An abstract of a dissertation is a summary and extraction of 
research work and contributions. Included in an abstract should be 
description of research topic and research objective, brief 
introduction to methodology and research process, and summarization 
of conclusion and contributions of the research. An abstract should be 
characterized by independence and clarity and carry identical 
information with the dissertation. It should be such that the general 
idea and major contributions of the dissertation are conveyed
without reading the dissertation.\par
     An abstract should be concise and to the point. It is a 
misunderstanding to make an abstract an outline of the dissertation and 
words “the first chapter”, “the second chapter” and the like should be 
avoided in the abstract.\par
     Key words are terms used in a dissertation for indexing, 
reflecting core information of the dissertation. An abstract may 
contain a maximum of 5 key words, with semicolons used in between to 
separate one another.
\end{enabstract}
%%%%%%%%%%%%%%%%%%%%%%%%%%%%%%%%%%%%%%

%--------------文章目录-------------
\tableofcontents
\listoffigures
%\addcontentsline{toc}{section}{插图清单}
\listoftables
%\addcontentsline{toc}{section}{附表清单}


%------------------------------------词汇------------------------------------------------------------
\begin{denotation}{2.5}{0}

\item[cluster] 集群
\item[Itanium] 安腾
\item[SMP] 对称多处理
\item[API] 应用程序编程接口
\item[PI]   聚酰亚胺
\item[劝  学] 君子曰:学不可以已。青,取之于蓝,而青于蓝;冰,水为之,而寒于水。
  木直中绳。(车柔)以为轮,其曲中规。虽有槁暴,不复挺者,(车柔)使之然也。故木
  受绳则直, 金就砺则利,君子博学而日参省乎己,则知明而行无过矣。吾尝终日而思
  矣,  不如须臾之所学也;吾尝(足齐)而望矣,不如登高之博见也。登高而招,臂非加
  长也,  而见者远;  顺风而呼,  声非加疾也,而闻者彰。假舆马者,非利足也,而致
  千里;假舟楫者,非能水也,而绝江河,  君子生非异也,善假于物也。积土成山,风雨
  兴焉;积水成渊,蛟龙生焉;积善成德,而神明自得,圣心备焉。故不积跬步,无以至千
  里;不积小流,无以成江海。骐骥一跃,不能十步;驽马十驾,功在不舍。锲而舍之,朽
  木不折;  锲而不舍,金石可镂。蚓无爪牙之利,筋骨之强,上食埃土,下饮黄泉,用心
  一也。蟹六跪而二螯,非蛇鳝之穴无可寄托者,用心躁也。\pozhehao{} 荀况
\end{denotation}

%%%%%%%%%%%%%%%%%%%%%%%%%%%%%%%%%%%%%%%

\pagenumbering{arabic}

\section{绪论}\label{ux7eeaux8bba}

\subsection{人脸识别研究现状}\label{ux4ebaux8138ux8bc6ux522bux7814ux7a76ux73b0ux72b6}

\subsection{行人再识别研究现状}\label{ux884cux4ebaux518dux8bc6ux522bux7814ux7a76ux73b0ux72b6}

\subsection{卷积神经网络}\label{ux5377ux79efux795eux7ecfux7f51ux7edc}

\subsection{论文组织结构}\label{ux8bbaux6587ux7ec4ux7ec7ux7ed3ux6784}

\section{卷积神经网络}\label{ux5377ux79efux795eux7ecfux7f51ux7edc-1}

\subsection{概述}\label{ux6982ux8ff0}

卷积神经网络(Convolutional Neural
Networks)是人工神经网络与深度学习思想的结合,是目前模式识别的研究热点。它的提出源于对猫的视觉皮层细胞的研究,1962年Hubel和Wiesel提出了感受野(receptive
field)的概念,1984年此概念在日本学者Fukushima提出的神经认知机(neocognitron)中首次被应用。神经认知机将一个视觉模式分解成许多子模式(特征)然后进入分层递阶式相连的特征平面进行处理,尝试在物体有位移或轻微变形的时候,也能完成识别。神经认知机可以看做第一个卷积神经网络的实现。
卷积神经网络的主要特点体现在两个方面:局部连接和权值共享。局部连接是指层与层神经元之间的连接采用局部连接代替全连接;权值共享是指同一层中神经元之间的连接权值是共享的。两者使卷积神经网络在很大程度上降低了参数数量,从而使网络的复杂度降低。由于其结构与生物神经网络非常相似,即使输入的图像不做任何预处理,卷积神经网络的识别效果也比较显著,同时避免了繁琐的特征提取的过程。
本章将详细介绍卷积神经网络的基本思想、拓扑结构、理论推导以及训练和测试。
\#\# 主要思想
根据Hubel和Wiesel对猫初级视皮层的研究,生物的初级视皮层包括简单细胞和复杂细胞,简单细胞主要负责感知其感受野内的特定边缘刺激,而复杂细胞则以简单细胞的输出为输入,并负责以更大的感受野来感受边缘刺激。根据简单细胞和复杂细胞的工作原理,卷积神经网络主要采用三种结构来进行视皮层的模拟:局部连接、权值共享以及子采样。
\#\#\# 局部连接
局部连接是指在相邻层之间不使用全连接而使用局部连接,从而不仅减少了需要训练的参数数量,而且利用了图像的局部特征信息。
如下图所示,图a为全连接,图b为局部连接。假设图片有1000
\$\times \$1000个像素的图片,有一百万的隐层神经元,全连接需要每一个隐层神经元连接到图像的每一个像素点,有100
\(\time\) 100 \(\times\) 1000000 =
\(10_{12}\)个连接,也就需要\(10_{12}\)个参数。局部连接则只需要每个节点只与其感受野中的像素点进行相连,假设其感受野为10\$\times\(10,则一百万个隐层神经元就只要10\)\times\(10\)\times\$100000
=
\(10_{8}\)个权值参数,权值参数的个数减少四个数量级。因此局部连接减少了所需训练的权值参数。

\subsubsection{权值共享}\label{ux6743ux503cux5171ux4eab}

\subsubsection{子采样}\label{ux5b50ux91c7ux6837}

\subsection{网络拓扑结构}\label{ux7f51ux7edcux62d3ux6251ux7ed3ux6784}

\subsubsection{卷积层}\label{ux5377ux79efux5c42}

\subsubsection{子采样层}\label{ux5b50ux91c7ux6837ux5c42}

\subsubsection{分类器}\label{ux5206ux7c7bux5668}

\subsection{理论推导}\label{ux7406ux8bbaux63a8ux5bfc}

\subsection{网络训练和测试}\label{ux7f51ux7edcux8badux7ec3ux548cux6d4bux8bd5}

\section{改进卷积神经网络}\label{ux6539ux8fdbux5377ux79efux795eux7ecfux7f51ux7edc}

\subsection{概述}\label{ux6982ux8ff0-1}

\subsection{主要思想}\label{ux4e3bux8981ux601dux60f3}

\subsection{网络拓扑结构}\label{ux7f51ux7edcux62d3ux6251ux7ed3ux6784-1}

\subsection{理论推导}\label{ux7406ux8bbaux63a8ux5bfc-1}

\section{基于改进卷积神经网络的人脸识别}\label{ux57faux4e8eux6539ux8fdbux5377ux79efux795eux7ecfux7f51ux7edcux7684ux4ebaux8138ux8bc6ux522b}

\section{基于改进卷积神经网络的行人再识别}\label{ux57faux4e8eux6539ux8fdbux5377ux79efux795eux7ecfux7f51ux7edcux7684ux884cux4ebaux518dux8bc6ux522b}

\section*{总结和展望}\label{ux603bux7ed3ux548cux5c55ux671b}
\addcontentsline{toc}{section}{总结和展望}

\hypertarget{refs}{}

%\include{chapters/summery}

%%%%%%%%%%%%%%%%%%%%%%%%%%%%%%%%%%%%%%%



%\include{chapters/appendix}  %%附录

\end{document}
