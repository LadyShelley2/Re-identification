\section{核鉴别稀疏保持嵌入}
\subsection{核鉴别信息权重}
将样本信息$X_{i}=[x_{i1},\ldots,x_{ik}]$映射到高维空间为$B_{i}=[\phi(x_{i1}),\ldots,\phi(x_{ik})]$,只要证明$\Vert x_{ij}-X_{i}t\Vert$越小,则$\Vert \phi(x_{ij})-X_{i}t\Vert$越小最小二乘问题变为
$$\min_{t}\Vert{\phi{(x_{ij})}-B_{i}t}\Vert_{2}$$
$$s.t.\quad It=1$$
因为$B$和$\phi{x}$是未知的,所以目标函数不能直接求解,所以将问题转化成以下约束:
$$\min_{t}\Vert{B^{T}\phi{(x_{ij})}-B^{T}B_{i}t}\Vert_{2}$$
$$s.t.\quad It=1$$
求证 $\Vert x_{ij}-X_{i}t\Vert$,$$\min_{t}\Vert{\phi{(x_{ij})}-B_{i}t}\Vert_{2}$$,$$\min_{t}\Vert{B^{T}\phi{(x_{ij})}-B^{T}B_{i}t}\Vert_{2}$$同时达到最小值

\begin{displaymath}
	B_{i}^{T}\phi{(x_{ij})}=
		\left[
			\begin{array}{ccc}
				\phi{(x_{i1})}\\
				\phi{(x_{i2})}\\
				\vdots\\
				\phi{(x_{ik})}
			\end{array}
		\right]\phi{(x_{ij})}=
		\left[
			\begin{array}{ccc}
				k(x_{i1},x_{ij})\\
				k(x_{i2},x_{ij})\\
				\vdots\\
				k(x_{ik},x_{ij})
			\end{array}
		\right]
\end{displaymath}

\begin{displaymath}
B_{i}^{T}B_{i}=
	\left[
		\begin{array}{ccc}
			\phi{(x_{i1})}\\
			\phi{(x_{i2})}\\
			\vdots\\
			\phi{(x_{ik})}
		\end{array}
	\right]
	\left[
		\begin{array}{cccc}
			\phi{(x_{i1})} & \phi{(x_{i2})} & \ldots & \phi{(x_{ik})}
		\end{array}
	\right]=
	\left[
		\begin{array}{cccc}
			k(x_{i1},x_{i1}) & k(x_{i1},x_{i2}) & \ldots & k(x_{i1},x_{ij})\\
			k(x_{i2},x_{i1}) & k(x_{i2},x_{i2}) & \ldots & k(x_{i2},x_{ij})\\
			\vdots & \vdots & \ddots & \vdots \\
			k(x_{i3},x_{i1}) & k(x_{i3},x_{i2}) & \ldots & k(x_{i3},x_{ij})
		\end{array}
	\right]
\end{displaymath}
目标函数转换成线性规划问题,可以求解。

\subsection{核稀疏重构权重}

令$C=[\phi{(X_{1})},\phi{(X_{2})},\ldots,\phi{(X_{c})}]$,令$C_{i}=[\phi{(X_{1})},\ldots,\phi{(X_{i-1})},\phi{(X_{i+1})},\ldots,\phi{(X_{c}})]$即从超完备字典中剔除样本$X_{i}$
令$er=\phi{(x_{ij})}-B_{i}\hat{t}=\phi{(x_{ij})}-C\hat{h}$,即样本由鉴别信息权重重构后得到的残差,对其对样本中的其他类进行重构:
\begin{displaymath}
	\begin{array}{ll}
		\min_{s} & \Vert{s}\Vert_{1}\\
		\\
		s.t.  & \Vert{er-C_{i}s}\Vert < \epsilon \\
		& Is = 0
	\end{array}
\end{displaymath}
同理由于函数$\phi$不能直接求解,将${er-C_{i}s} $左乘$C^{T}$转换成${C_{i}^{T}(er-C_{i}s)}$,即$C^{T}\phi{(x_{ij})}-C^{T}C\hat{h}-C^{T}C_{i}s$

求证
\newtheorem{confirm}{公理}[section]
	\begin{confirm}
		对任意$\epsilon\ge 0$都存在$\delta\ge 0$,只要$C^{T}\phi{(x_{ij})}-C^{T}C\hat{h}-C^{T}C_{i}s<\delta$
		就有$er-C_{i}s < \epsilon$.
	\end{confirm}

\begin{displaymath}
	C^{T}\phi{(x_{ij})}=
		\left[
			\begin{array}{cccc}
				\phi{(x_{11})} & \phi{(x_{12})} & \ldots & \phi{(x_{1k})}\\
				\phi{(x_{21})} & \phi{(x_{22})} & \ldots & \phi{(x_{2k})}\\
				\vdots & \vdots & \ddots & \vdots\\
				\phi{(x_{c1})} & \phi{(x_{c2})} & \ldots & \phi{(x_{ck})}
			\end{array}
		\right]\phi{(x_{ij})}
		=\left[
			\begin{array}{cccc}
			  k(x_{11},x_{ij}) & k(x_{12},x_{ij}) & \ldots &\k(x_{1k},x_{ij}) \\
			  k(x_{21},x_{ij}) & k(x_{22},x_{ij}) & \ldots &\k(x_{2k},x_{ij}) \\
			  \vdots & \vdots & \ddots & \vdots \\
			  k(x_{c1},x_{ij}) & k(x_{c2},x_{ij}) & \ldots &\k(x_{ck},x_{ij}) 
			\end{array}
		 \right]
\end{displaymath}
\begin{displaymath}
	C^{T}C=
		\left[
			\begin{array}{cccc}
				\sum_{j=1}^{k}k(x_{1j},x_{1j}) & \sum_{j=1}^{k}k(x_{1j},x_{2j}) & \ldots & \sum_{j=1}^{k}k(x_{1j},x_{cj}) \\
				\sum_{j=1}^{k}k(x_{2j},x_{1j}) & \sum_{j=1}^{k}k(x_{2j},x_{2j}) & \ldots & \sum_{j=1}^{k}k(x_{2j},x_{cj}) \\
				\vdots & \vdots & \vdots & \ddots \\
				\sum_{j=1}^{k}k(x_{cj},x_{1j}) & \sum_{j=1}^{k}k(x_{cj},x_{2j}) & \ldots & \sum_{j=1}^{k}k(x_{cj},x_{cj}) 
			\end{array}
		\right]
\end{displaymath}
\begin{displaymath}
	C^{T}C_{i}=
		\left[
			\begin{array}{cccccc}
				\sum_{j=1}^{k}k(x_{1j},x_{1j}) & \ldots & \sum_{j=1}^{k}k(x_{1j},x_{i-1,j}) & \sum_{j=1}^{k}k(x_{1j},x_{i+1,j}) &\ldots & \sum_{j=1}^{k}k(x_{1j},x_{cj}) \\
				\sum_{j=1}^{k}k(x_{2j},x_{1j}) & \ldots & \sum_{j=1}^{k}k(x_{2j},x_{i-1,j}) & \sum_{j=1}^{k}k(x_{1j},x_{i+1,j}) &\ldots & \sum_{j=1}^{k}k(x_{2j},x_{cj}) \\
				\vdots & \ddots & \vdots & \vdots & \ddots & \vdots \\
				\sum_{j=1}^{k}k(x_{cj},x_{1j}) &\ldots & \sum_{j=1}^{k}k(x_{cj},x_{i-1,j}) & \sum_{j=1}^{k}k(x_{cj},x_{i+1,j}) &\ldots & \sum_{j=1}^{k}k(x_{cj},x_{cj}) 
			\end{array}
		\right]
\end{displaymath}
将$\hat{s}$按照不同类别分块,得$\hat{s}=[\hat{s}_{1},\ldots,\hat{s}_{i-1},\hat{s}_{i+1},\ldots,\hat{s}_{c}]$,将$\hat{s}$稀疏到完备字典上得$\hat{l}=[\hat{s}_{1},\ldots,\hat{s}_{i-1},\vec{0},\hat{s}_{i+1},\ldots,\hat{s}_{c}]$ 结合得到的鉴别权重$\hat{t}$,将系数串联得到$\hat{d}=\hat{l}+\hat{h}$,这样我们就得到了核鉴别稀疏权重。由于$I\hat{l}=I\hat{s}=0$,$I\hat{t}=1$,故$I\hat{d}=1$,能够保持旋转,平移,尺度不变的特性,证明过程见\cite{马小虎2014基于鉴别稀疏保持嵌入的人脸识别算法}

\subsection{KDSPE的目标函数}
与DSPE同理,KDSPE的目标函数数学模型如下:
$$\min_{W}\quad\sum_{i=1}^{n}\Vert{W^{T}x_{i}-W^{T}X\hat{d}_{i}}\Vert^{2}$$
$$s.t.\quad W^{T}XX^{T}W = 1$$
与DSPE的推导相似,可得到
$$XMX^{T}W = \lambda XX^{T}W$$
选取得到的最大的$d$个特征值所对应的特征向量$\mathbf{a}_{i}$构成的特征子空间,即可得到KDPE的线性降维映射$W_{DSPE}=[a_{1},a_{2},\ldots,a_{d}]$.