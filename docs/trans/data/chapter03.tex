\subsection{KISS度量学习}
我们的方法考虑到两个独立的                    ,相同种类与否由\cite{ma2014covariance}是否属于该种类的可信度决定。从数学推导的视角来看,最优的数据上的决策是否一对点对相同可以由一个相似度来获得。所以我们检验假说H0两者不同和对立假说H1 两者相同:
$$\delta(x_{i},y_{j}) = \log\bigg(\frac{p(x_{i},x_{j}|H_{0})}{p(x_{i},x_{j}|H_{1})}\bigg)$$

获得较大的值标识H0是有效的。相反,较小的值意味着被假说H0拒绝,点对被看做是相同的。为了使特征空间中的实际分布具有独立性,我们假设点对间的差异为0,所以等式9可以被重写为
$$\delta(\mathbf x_{ij}) = \log \bigg(\frac{p(\mathbf x_{ij}|H_{0})}{p(\mathbf x_{ij}|H_{1})}\bigg) = \log\bigg(\frac{f(\mathbf x_{ij}|\theta_{0})}{f(\mathbf x_{ij}|\theta_{1})}\bigg)$$
其中   是假说H1点对相同的含有参数  的概率密度函数,与假设点对不相同的假说H0相对。将特征间的差异空间看作一个高斯结构我们可以简化这个问题,等式被重新写为
$$\delta(\mathbf x_{ij}) = \log \Bigg(\frac{\frac{1}{\sqrt{2\pi|\Sigma_{y_{ij}=0}|}}\exp(-1/2 \mathbf x_{ij}^T \Sigma_{y_{ij}=0}^{-1}\mathbf x_{ij}}{\frac{1}{\sqrt{2 \pi |\Sigma_{y_{ij}=1}|}}\exp(-1/2 \mathbf x_{ij}^{T}\Sigma_{y_{ij}=1}^{-1} \mathbf x_{ij})}\Bigg)$$

其中
$$\Sigma_{y_{ij}=1} = \sum_{y_{ij} = 1}(\mathbf x_{i} - \mathbf x_{j})(\mathbf x_{i} - \mathbf x_{j})^{T}$$

$$\Sigma_{y_{ij}=0} = \sum_{y_{ij} = 0}(\mathbf x_{i} - \mathbf x_{j})(\mathbf x_{i} - \mathbf x_{j})^{T}$$

点对间的差异是对称的。所以,我们有均值为0,且公式  。高斯中最大化相似度评估等价于采用最小二乘法最小化马氏距离从。 这是我们可以分别针对两个独立的集找出个字相关的方向。通过取对数,我们可以重新整理相似度的检验如下:
$$\delta(x_{ij})= \mathbf x_{ij}^{T}\Sigma_{y_{ij}=1}^{-1}\mathbf x_{ij} + \log(|\Sigma_{y_{ij}=1}|) - \mathbf x_{ij}^{T}\Sigma_{y_{ij}=0}^{-1}\mathbf x_{ij} - \log(|\Sigma_{y_{ij}=0}|)$$

更进一步地,我们去掉常数项因为他们只是提供了一个偏移量,简化如下:
$$\delta (\mathbf x_{ij})=\mathbf x_{ij}^{T}(\Sigma_{y_{ij}=1}^{-1}-\Sigma_{y_{ij}=0}^{-1})\mathbf x_{ij}$$


最后,我们获得我们的可以反映了相似度检验的马氏距离度量如下

$$d_{\mathbf M}^2(\mathbf x_{i} - \mathbf x_{j}) = (\mathbf x_{i} - \mathbf x_{j})^{T} \mathbf M (x_{i}-x_{j})$$
通过重新映射     到半正定矩阵。所以,我们通过等价裁剪  的范围我们得到了   。

