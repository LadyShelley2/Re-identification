\section{前言}
距离或相似度的度量学习是机器学习中非常重要急迫的领域,在计算机视觉中有各种各样的应用。它可以明显提高跟踪,图像检索,人脸识别,聚类或者行人再识别。度量学习算法的目标是利用更加简单更加通用的相似度方法

在很多机器学习问题中展示出非常好的普遍性的一种经典距离函数是马氏度量学习。其目标是寻找一个全局的特征空间先行转换,从而使相关的维数被强调而不相关的被丢弃。因为在马氏度量和多变量高斯函数中存在双射,我们可以想到与之相关的协方差矩阵。矩阵通过任意的线性旋转和比例变换适应所要求的几何。映射之后采用浅显易懂的欧式距离进行测量。

学习马氏度量的机器学习算法最近在计算机视觉领域吸引了大量的兴趣。包括大边缘最近邻学习(LMNN),信息论度量学习(ITML),和被认为最高水平的逻辑斯蒂判别式度量学习(LDML)。LMNN目标在于提高K-nn分类。它为每个实例建立了一个局部的周界,这个周界包围着相同标记(目标近邻)的K最近邻实例,加上一个边缘。为了减少侵入周界不同标识实例(冒充目标)的数量,度量方法被迭代使用。通过加强目标近邻间的联系同时减弱与冒充者的联系。概念上听起来,LMNN有时候比较容易由于对数据缺乏规则化产生过拟合的现象。David通过明确集成了一个数据规则化的步骤来避免过拟合。他们的公式在满足距离函数所给约束同时最小化与先前优化的距离度量区域的差异熵。Guillaumin介绍了一个马氏距离的概率视角,在其中一个后验型种类概率被看成成相同(不同)的方法。因此,他们的目标在于迭代地适应马氏度量从而使对数似然函数的值最大。后验概率是通过sigmoid 函数建模的,sigmoid函数反映了实例如果距离在某个阈值之下共享标识的事实。原则上,所有的这些方法都可以在没有见过的数据上概括得很好。他们专注在强健的损失函数,规则化解决方案从而避免过拟合。

考虑到数据数量不断增长,在大规模数据集上学习马氏度量对可扩展性和监督所需要的阶数提出了更高的要求。时常对所有数据点进行具体的完全监督标识是不可行的,而通过等价约束的形式更容易进行具体标识。在一些实例中,采用这种自动弱监督的方式会更加可能。例如跟踪一个对象。所以在大规模数据集上实施将会面对额外的等价转换和等价约束挑战。

为了满足这些要求,我们研究了一个基于等价约束的有效度量方法。他们建立点与点间基本关系,被认为是距离肚量学习算法中的类似相似度函数的输入。我们的方法被一个基于可能性测试的数学推断所驱动。我们展示了尺度衡量的结果不易于过拟合而且可以非常高效地获得。和其他方法相比,我们不依赖于反复的迭代优化过程。然而,我们的方法对大数据集来说是可扩展的,因为他仅仅包括了两个小型且大小固定的协方差的计算。和KISS原则相似,我们是我们的方法每一步设计简单且有效所以称之为简单有效的度量方法(KISS metric)。

我们在不同的实验样本中证明我们的方法可以达到甚至优于目前最先进的尺度度量方法,同时在训练阶段有数量级倍的提高。尤其是我们提供了在两个最近的人脸识别样本上的结果。其不明显的特征和在姿势、灯光以及面部表情上的变化是学习算法中的一个挑战。更进一步地,我们研究了不连续空域摄像头下的行人再识别课题以及以玩具车为例对未见过对象实例的对比。在VIPeR和玩具车数据集中,我们水平提高到甚至超过了目前领域中最为先进的水平。而且,在LFW,我们采用标准SIFT特征得到了目前最好的结果。

文章的剩余部分是如下组织的。在第二部分中我们讨论了支持我们研究的相关尺度衡量算法。进一步,我们在第三部分中介绍了我们的KISS尺度学习方法。扩展的实验和性能评估以及扩展性评估在第四部分中展示。最后,我们在第五部分中得出结论并总结了这篇论文。
