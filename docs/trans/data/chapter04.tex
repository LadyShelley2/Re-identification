\section{实验}
为了展示我们方法的应用广泛性,我们在含有多种特征的不同样本数据集上测试了我们的方法。我们实验的目标主要分为两点:第一,我们想展示我们的方法是也可以概括未知数据,甚至比目前最先进的算法效果还要好。第二,我们想要证明我们的运算速度将会产生数量级上的提高。这对大规模和在线应用都有明显的好处。在4.1部分,我们首先测试了无约束环境下人脸识别。接下来,在4.2部分,我们通过空域不连接的摄像头研究了行人再识别问题。最后,在4.3部分,我们在玩具车数据集上测试和评估了我们的算法,意在比较对未知物体的识别。

为了和其他尺度学习方法进行对比,我们收集了原始代码并且采用相同的输入数据。代码由各自的作者分别提供。更进一步地,我们将我们的方法和相关的区域一定的方法进行对比。并且画出了各自的EER曲线
\subsection{人脸识别}
以下部分,我们在两个具有挑战性的人脸识别数据集上证明了我们方法的性能,分别是Labeled Faces in the Wild (LFW)\cite{huang2008labeled} 和Public Figures Face Database (PubFig)\cite{kumar2009attribute}。据此,对人脸识别的研究被分为两个不同的目标:人脸识别(识别一张人脸)和人脸验证(判断两张人脸图片是否是相同的个体)。人脸识别自身要求有大量的带有注解的个人人脸,真实世界中的数据库往往不能满足。相反,人脸验证需要较少的注释而且可以在大规模上也能实现。所以,根据以上分析,我们更加关注人脸验证课题。

\subsubsection{Wild数据集中的被标识人脸}
Wild数据集\cite{huang2008labeled}中被标识人脸共包含了5749个人的13233张无约束人脸,可以认为是当前最高水平的人脸识别样本库。这个数据库被认为非常具有挑战性,因为它在姿势、亮度、面部表情、年龄、性别、种族和总体摄影以及环境状况等方面的有多种多样的变化。在图表2(a)中我们给出了一些说明性的例子。LFW的一个重要特征是数据库中的主体在数据库的任何一个部分中都是和其他主体互相排斥的。所以,对于人脸验证课题来讲,可以根据在训练中未出现过的个体进行测试。

数据被分为十个部分以便进行交叉验证。每个部分含有300对相同和300对不同的数据。结论的分数在10个部分上被平均。在受约束的协议下只有对相同的组或不同的组进行等价约束变化,不能根据主体的身份进行推算。例如,取样更多的训练数据是允许的。

在我们的实验中,我们采取了Guillaumin\cite{guillaumin2009you}的人脸表示法。这种方法主要在三个不同尺度上提取了9个利用SIFT算法\cite{lowe2004distinctive}自动检测到的面部标志。最终的描述结果是一个3456维的向量。为了使它在距离尺度学习算法中可跟踪,我们利用PCA将之降维为100维的子空间。为了评估不同的尺度衡量算法,有一个公正的比较评估,我们利用相同的特征和PCA得出的维数训练分类器。PCA的影响不是非常重要。利用不同的降维算法对所有测试的算法得出的结论表示在性能上没有很大的差异。但是在线性支持向量机上,我们直接对人脸描述子进行训练得出了更好的结果。

在图表2(b)我们展示了LDML\cite{guillaumin2009you},ITML\cite{davis2007information},LMNN\cite{weinberger2009distance},SVM\cite{chang2011libsvm}和我们的方法KISSME的受试者工作特征曲线以及以欧氏距离为基线的相同对的马氏距离。需要特别说明的是在LMNN中我们基于实际分类信息提供了更多的有监督训练,因为它需要三元组。

相同分类样本对($\mathbf M_{y=1}$)的马氏距离比欧式距离的性能好。马氏距离性能提高了大约$7\%$。有趣的是,LMNN不能概括其他尺度上的额外信息。KISSME比其他尺度度量方法稍微好一些。它达到了$80.5\%$的等错误率,是目前这种样本类型中效果最好的。当然,最近在LFW上最先进de方法提供了更好的结果但是同时它需要了更多领域的知识,因为这些方法关注仅仅在人脸上。

当分析表格1中给出的训练次数是,我们方法最主要的优势也很明显。事实上,与LMNN,ITML和LDML相比,我们的方法在计算上更加高效,尽管如此,仍然可以输出非常具有竞争力的结果。
\renewcommand\tablename{表格}
\begin{table}[!hbp]
\centering
\renewcommand\arraystretch{1.5}
\begin{tabular}{|c|c|c|c|c|}
\hline
 \textbf{Method}  & \textbf{LFW} & \textbf{PubFig} & \textbf{VIPeR} &  \textbf{ToyCars} \\
 \hline
 \textbf{KISSME } & 0.05s & 0.07s & 0.01s & 0.04s \\
 \hline
 \textbf{SVM}  & 12.78s & 0.84s & - & 0.60s \\
 \hline
 \textbf{ITML} & 24.81s & 20.82s & 8.60s & 14.05s \\
 \hline
 \textbf{LDML} & 307.23s & 2869.91s & 0.72s & 1.21s \\
 \hline
 \textbf{LMNN} & 1198.69s & 783.66s & 27.56s & 0.79s \\
 \hline
\end{tabular}
\caption{平均训练时间。LDML,ITML采用多线程,在24核,3.06GHz Xeon上测试}
\end{table}
\subsubsection{Public Figures人脸图片数据库}
PubFig 数据集\cite{kumar2009attribute}和LFW有很多的共同点。他也是一个非常具有挑战性的大规模,真实的数据库,包含了200个人的58797张图片,这些图片从谷歌图片和FlickR上手机。人脸验证的样本包括了10个交叉验证文件夹,其其中包括了1000对内部人员和1000对外部人员的信息。每个文件夹中的信息从14个个人中获得。和LFW相同,出现在测试中的人脸在训练过程中并未出现过。

在数据库中比较有趣的是一种被称为“高水平”特征被用来描述人脸视觉特征的出现和消失。容貌被用一些可以可命名的属性例如性别、种族、年龄、头发等或者和面部区域相似度有关联的“微笑”等信息来编码对应确切代表的一个人。这种间接的描述方式输出的很好的属性,例如和低水平的特征相比有一定程度的鲁棒性提升。而且,它给我们在评估距离尺度学习算法的评估上也提供了补足的特征类型。

在图表4中我们画出了LDML\cite{guillaumin2009you}、ITML\cite{davis2007information}、LMNN\cite{weinberger2009distance}、SVM\cite{chang2011libsvm}和KISSME的ROC曲线以及两条极限。可以看出我们的方法比LDML、ITML、LMNN表现出色,达到了目前最先进的性能——基于Kumar的方法\cite{kumar2009attribute}的SVM。LDML我们的算法达到了相同的结果然而却在训练时间上慢了数量级倍。这使该方法在在线计算和大规模计算中变得不现实。有趣的是,ITML的性能下降到甚至普通欧氏距离水平以下。在图表2中我们也阐述了LFW的属性特征的性能。

\subsection{行人再识别}

VIPeR数据集\cite{gray2007evaluating}包括632对两个不同摄像头视角下在户外拍摄的个人图片。这些低分辨率的图片在姿势,视角有明显的变化,并且在亮度上也有很大的不同,例如有高亮和阴影的部分。示例中大多数的图片对包含九十度以上的视角转化,使行人再识别很具挑战性。在图3中给出了一些示例。为了将我们的方法和其他方法作比较,我们采用了文献\cite{farenzena2010person,gray2008viewpoint}中所定义的评价协议。作者将632张的图片集随机分成两组316张的图片集,一组用于训练,另外一组用于测试,在多次实验之后计算平均值。由于没有预先确定图片集和如何使两组得到不同的图片对的规则。因此,我们根据随机组合不同人的图片生成不同的图片对。

为了表示图片,我们编译了一个极其简单的描述子。首先,我们将图片按照8*16分为8行8列的有重叠部分的块状。然后,为了描述图片的颜色信息,我们了HSV和Lab直方图,每个通道含有24个颜色小区间。第三,我们利用LBPs捕捉纹理信息。最后,对于距离尺度学习方法,我们将连接起来的描述子通过PCA映射到一个34维的子空间。

为了展示各种算法的性能,我们展示了累计匹配特征曲线(CMC)\cite{wang2007shape}。它们表达了在前n级项正确匹配的期望。为了得到一个合理的数据,我们将100次运行进行平均。在图3中,我们展示了不同尺度学习算法的累计匹配特征曲线,并且在表2中展示了我们方法与目前最先进的方法\cite{dikmen2010pedestrian,farenzena2010person,hirzer2011person,zheng2011person}在前50级命中的性能对比结果。如我们可见,我们在所有级别均得到了有力的结果。我们的方法比其他方法\cite{farenzena2010person,gray2008viewpoint,prosser2010person}性能更好,尽管和他们相比我们并没有使用前景-背景分离处理。与此同时,我们在计算上更加高效。
\renewcommand\tablename{表格}
\begin{table}[!hbp]
\centering
\renewcommand\arraystretch{1.5}
\begin{tabular}{|c|c|c|c|c|}
\hline
  \textbf{RANK}  & \textbf{1} & \textbf{10} & \textbf{25} &  \textbf{50} \\
 \hline
 \textbf{KISSME}  & \textbf{19.6}\%s & \textbf{62.2}\% & \textbf{80.7}\% & \textbf{91.8}\% \\
 \hline
 \textbf{LMNN}  & 19.0\% & 58.1\% & 76.9\% & 89.6\% \\
 \hline
 \textbf{ITML} & 15.2\% & 53.3\% & 74.7\% & 88.8\% \\
 \hline
 \textbf{LDML} & 10.4\% & 31.3\% & 44.6\% & 60.4\% \\
 \hline
 $\mathbf M_{y=1}$ & 16.8\% & 50.9\% & 68.7\% & 82.0\%  \\
 \hline
 \textbf{L2}  & 10.6\% & 31.8\% & 44.9\% & 60.8\%  \\
 \hline
\end{tabular}
\caption{平均训练结果(在VIPeR上100次测试结果平均)}
\end{table}
\renewcommand\tablename{表格}
\begin{table}[!hbp]
\centering
\renewcommand\arraystretch{1.5}
\begin{tabular}{|c|c|c|c|c|}
\hline
 \textbf{RANK}  & \textbf{1} & \textbf{10} & \textbf{25} &  \textbf{50} \\
 \hline
 \textbf{KISSME}  & \textbf{20}\% & \textbf{62}\% & \textbf{81}\% & \textbf{92}\% \\
 \hline
 \textbf{SDALF}\cite{farenzena2010person} & \textbf{20}\% & 50\% & 70\% & 85\% \\
 \hline
 \textbf{DDC}\cite{hirzer2011person} & 19\% & 52\% & 69\% & 80\% \\
 \hline
 \textbf{PRDC}\cite{zheng2011person} & 16\% & 54\% & 76\% & 87\% \\
 \hline
 $\mathbf {KISSME}^*$& \textbf{22}\% & \textbf{68}\% & \textbf{85}\% & \textbf{93}\%  \\
 \hline
 $\mathbf {LMNN-R}^*$\cite{dikmen2010pedestrian} & 20\% & \textbf{68}\% & 84\% & 93\%  \\
 \hline
\end{tabular}
\caption{目前最先进方法一览表}
\end{table}

\subsection{玩具车数据集}
LEAR ToyCars数据集\cite{nowak2007learning}包含了14种玩具小汽车和玩具卡车,每种256张图片截图。数据集在姿势,光照和凌乱的背景上有很大的不同。实验目的是根据已知种类的汽车来对之前未见过的物体实例进行对比(如图5所示)。所以,此实验是为了分类是否一组图片展示的是同一物体。训练数据集包含了7种物体的相关的1185张相同和7330张不同的图片组。剩余的7个物体的图片示例用作测试。由于图片在水平方向上的像素不同,我们将其补零使之具有一个规范的图像大小。

我们采用行人再识别中所使用的方法进行图像信息提取和标识。所以,图像被分为30*30的无重叠的块状。我们利用HSV和Lab提取颜色信息,利用LBPs\cite{ojala2002multiresolution}记录纹理信息。整体图片的描述子是由块状描述子组合而成。利用PCA将描述子映射到一个50维的子空间。

我们将在这个数据集上的实验结果与Nowak和Jurie\cite{nowak2007learning}最新的方法进行了对比,他们的方法基于一个极其随机的树结构的集合体。集合体通过使匹配成对图片的相关联补丁输出相似的结果来量化相关联补丁对的不同。相关联的补丁通过NCC定位到一个局部的邻近区域。在测试中,两张图片对的相似度是在同一叶子结点结束的相关补丁的加权和。

在图6中,我们对比了我们的方法、Nowark和Jurie\cite{nowak2007learning}的方法和其他尺度学习方法的ROC曲线。然后,我们提供了一个标准线性支持向量机下的基准线。利用支持向量机\cite{chang2011libsvm}输出的等错误率为$82\%$,是一个合乎情理的结果。有趣的是,一些尺度学习算法在欧氏距离上进行提高。只有LMNN和SVM性能相近。利用通过正确的图片学习到的马氏距离,我们达到了$89.8\%$的等错误率,比Nowak和Jurie的方法性能好。KISSME大大地将算法性能增强,提高至$93.5\%$。如果有人考虑17小时的计算时间\cite{nowak2007learning},我们的方法再一次展示了他有效性和高效性的优势。

